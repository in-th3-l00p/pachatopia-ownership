\documentclass{article}

\usepackage{parskip}

\title{pachatopia.org whitepapaer}
\author{Cătălin Tișcă}

\begin{document}

\maketitle

\section{introduction}
\textbf{Pachatopia} is an association that creates nature reserves and promotes 
sustainable agriculture and eco-responsible living, inspired by Indigenous respect for the Earth.

Pachatopia is building a \textit{blockchain-based map} to track its land use, allowing donors to sponsor 
land parcels and receive on-chain rights to their produce, community membership, and educational 
access—advancing ReFi accountability while laying the groundwork for a genetic bank preserving the 
region's unique plant biodiversity.

\section{technical implementation}

\subsection{design}

\subsection{web2 details}
Pretty much the frontend and the backend concerns of the app. A backend is still required, as it is too
early to go for a decentralized storage infrastructure like \textit{IPFS}

\subsubsection{tech stack}
\begin{itemize}
    \item \textbf{Vite + React} - fully open source frontend library, universal choice
    \item \textbf{TailwindCSS + shadcn} - allows enough customization \& flexibility for getting the UI done right
    \item \textbf{ConvexDB \& @convex/auth} - prolly the best option nowadays
    \item \textbf{react-leaflet} - React integration
    \item \textbf{leaflet-draw} - admin polygon drawing/editing
    \item \textbf{turf.js} - geospatial calculations (area in m², parcel splitting, overlap detection)
\end{itemize}

\subsection{blockchain infra}
KISS \textit{(keep it simple stupid)}, everything on the ethereum chain can be done quite easily.
\begin{itemize}
    \item \textbf{ERC-721} - each m² is unique, this is the perfect standard to identify each one of them
    \item \textbf{Guild.xyz} - the gatekeeper, provides an \textit{identity portal}
    \item \textbf{Snapshot} - gasless voting, good for establishing the \textit{DAO}
\end{itemize}

\end{document}